% !Mode:: "TeX:UTF-8" 
\renewcommand{\textfraction}{0.05} 
\BiChapter{实验要求}{}

% -------------------------------章节分割线-------------------------------
\BiChapter{在IntelliJ中配置代码审查与分析工具}{}
\BiSection{Checkstyle}{}
\BiSection{PMD}{}
\BiSection{FindBugs}{}
\BiSection{VisualVM}{}

% -------------------------------章节分割线-------------------------------
\BiChapter{本次实验所评审的代码}{}


\noindent 姓名:张冠华~\\
学号:1150310323~\\
Github地址:https://github.com/miuws~\\

\noindent 姓名:王珊~\\
学号:1150310302~\\
Github地址:https://github.com/arthua196
\begin{figure}[h]
\begin{center}
  \includegraphics[width=\linewidth]{project.jpg}
  项目清单
\end{center}
\end{figure}

\begin{figure}[h]
\begin{center}
  \includegraphics[width=\linewidth]{src.jpg}
  源码清单
\end{center}
\end{figure}


% -------------------------------章节分割线-------------------------------
\BiChapter{代码review记录}{}
\begin{adjustwidth}{-1pt}{}
\begin{tabular}{|c|c|c|c|}
\hline
 问题描述 & 类型 & 所在代码行号 &  修改方式\\
\hline
\makecell[l] {在点击退出按钮后 \\ GUI界面关闭 \\ 但程序仍在运行} & 
\makecell[l] {程序退出控制}  & 
\makecell[l] {MainPage.java \\ : 467} &
\makecell[l] {增加代码 \\ setDefaultCloseOperation \\ (WindowConstants. \\ EXIT\_ON\_CLOSE) }\\ 

\hline
\makecell[l] {没有对Graphviz \\ 关键字特别处理 \\ 展示图失败} & 
\makecell[l] {外部方法调用}  & 
\makecell[l] {TextMaker.java \\ : 168} &
\makecell[l] {使用引号 \\ 包围所有关键词 }\\ 

\hline
\makecell[l] {对文本逐字符处理 \\ 效率极低} & 
\makecell[l] {运行效率}  & 
\makecell[l] {TextMaker.java \\ : 53} &
\makecell[l] {使用一行正则表达式 \\ 完成文本处理} \\ 

\hline
\makecell[l] {重新打开对话框 \\ 以前输入的文字混乱} & 
\makecell[l] {GUI界面管理}  & 
\makecell[l] {MainPage.java \\ : 322} &
\makecell[l] {每次打开对话框 \\ 初始化新的对话框} \\ 


\hline
\makecell[l] {使用 \\ HIDE\_ON\_CLOSE \\ 关闭对话框} & 
\makecell[l] {GUI资源释放}  & 
\makecell[l] {MainPage.java \\ : 365} &
\makecell[l] {改为 \\ DISPOSE\_ON\_CLOSE} \\ 
\hline

\end{tabular}
\end{adjustwidth}


% -------------------------------章节分割线-------------------------------
\BiChapter{Checkstyle所发现的代码问题清单及原因分析}{}
(使用 Sun Checks 规则)
~\\

\begin{adjustwidth}{-1pt}{}
\begin{tabular}{|c|c|c|c|c|}
\hline
编号 & 问题描述 & 类型 & 所在代码行号 & 修改策略 \\
\hline
1 &
\makecell[l] {类缺少JavaDoc} &
\makecell[l] {文档缺失} &
\makecell[l] {Edge.java \\ :3} &
\makecell[l] {补充Edge类文档} \\

\hline
2 &
\makecell[l] {大括号应位于 \\ 类、方法定义同一行} &
\makecell[l] {类、方法 \\ 定义格式} &
\makecell[l] {Edge.java \\ :4} &
\makecell[l] {将大括号放在 \\ 类、方法定义同一行} \\

\hline
3 &
\makecell[l] {','前缺少空格} &
\makecell[l] {空格格式} &
\makecell[l] {Edge.java \\ :5} &
\makecell[l] {补充空格} \\

\hline
4 &
\makecell[l] {应避免在 \\ 字表达式中赋值} &
\makecell[l] {赋值格式} &
\makecell[l] {Edge.java \\ :10} &
\makecell[l] {将赋值拆分成多行} \\

\hline
5 &
\makecell[l] {参数xxx应 \\ 定义为final的} &
\makecell[l] {只读参数、变量 \\ 用法} &
\makecell[l] {Edge.java \\ :12} &
\makecell[l] {为变量声明 \\ 增加final关键字} \\

\hline
6 &
\makecell[l] {'\{'后应换行} &
\makecell[l] {避免使用一行 \\ 定义方法} &
\makecell[l] {Edge.java \\ :33} &
\makecell[l] {为方法定义 \\ 规范换行} \\

\hline
7 &
\makecell[l] {if/else结构 \\ 必须使用大括号} &
\makecell[l] {控制结构 \\ 可读性} &
\makecell[l] {Edge.java \\ :36} &
\makecell[l] {为if结构 \\ 添加大括号} \\

\hline
8 &
\makecell[l] {数组大括号 \\ 位置错误} &
\makecell[l] {数组定义规范} &
\makecell[l] {Graph.java \\ :121} &
\makecell[l] {数组中括号 \\ 移至变量类型后} \\

\hline
9 &
\makecell[l] {xxx是一个 \\ 魔术数字} &
\makecell[l] {直接常数} &
\makecell[l] {Graph.java \\ :135} &
\makecell[l] {将数字赋值给常量} \\

\hline
10 &
\makecell[l] {不应以.*形式导入xxx} &
\makecell[l] {import规范} &
\makecell[l] {MainPage.java \\ :1} &
\makecell[l] {GUI使用了大部分 \\ 包中的类,不修改} \\

\hline
11 &
\makecell[l] {xxx应为private \\ 并配置访问方法} &
\makecell[l] {访问权限规范} &
\makecell[l] {MainPage.java \\ :13} &
\makecell[l] {访问权限改为private \\ 添加访问方法} \\

\hline
12 &
\makecell[l] {名称必须匹配 \\ 表达式xxx} &
\makecell[l] {命名规范} &
\makecell[l] {MainPage.java \\ :22} &
\makecell[l] {refactor修改命名} \\

\hline
13 &
\makecell[l] {本行字符数xxx \\ 最多xxx} &
\makecell[l] {行长度规范} &
\makecell[l] {MainPage.java \\ :80} &
\makecell[l] {拆成多行} \\
\hline

\end{tabular}
\end{adjustwidth}
~\\
(使用 Google 规则集的不同之处)
~\\
\begin{adjustwidth}{-1pt}{}
\begin{tabular}{|c|c|c|c|c|}
\hline
编号 & 问题描述 & 类型 & 所在代码行号 & 修改策略 \\
\hline
1 &
\makecell[l] {缩进空格应为两个} &
\makecell[l] {缩进格式} &
\makecell[l] {TextMaker.java \\ :13} &
\makecell[l] {(和规则集有关,不修改)} \\

\hline
2 &
\makecell[l] {包名导入顺序错误} &
\makecell[l] {import顺序错误} &
\makecell[l] {MainPage.java \\ :3} &
\makecell[l] {更改包导入顺序} \\

\hline
3 &
\makecell[l] {注释中的空行 \\  应该在<p>标签后} &
\makecell[l] {缩进格式} &
\makecell[l] {Graphviz.java \\ :23} &
\makecell[l] {(和规则集有关,不修改)} \\

\hline
4 &
\makecell[l] {其它if,for等 \\ 缩进空格数} &
\makecell[l] {缩进格式} &
\makecell[l] {Graph.java \\ :35} &
\makecell[l] {(和规则集有关,不修改)} \\
\hline
\end{tabular}
\end{adjustwidth}

~\\~\\
\noindent 小结:~\\
Sun和Google规则集大致相同。它们主要对这些问题进行检查:~\\
1、缩进~\\
2、有利于可读性的空格~\\
3、代码块是否正确地被大括号包围~\\

\noindent 不同的之处有:~\\
1、Google对缩进的要求是Sun规则集的一般~\\
2、Google规则集对JavaDoc的格式检查更严格~\\
3、Google规则集检查包名的导入顺序~\\
4、Sun规则集会检查部分内部逻辑


% -------------------------------章节分割线-------------------------------
\BiChapter{PMD所发现的代码问题清单及原因分析}{}
优先级按照 https://pmd.github.io/pmd-5.8.1/pmd-java/rules/java 文档定义
~\\
\begin{adjustwidth}{-2em}{}
\begin{tabular}{|c|c|c|c|c|}
\hline
优先级 & 问题描述 & 违反的规则集合 & 代码行号 & 修改策略 \\
\hline
3 & 
\makecell[l] {变量、参数名过短 \\ 不易于理解} & 
\makecell[l] {naming} &
\makecell[l] {Edge.java \\ : 5} &
\makecell[l] {使用refactor \\ 更改命名} \\

\hline
3 & 
\makecell[l] {缺少包定义} & 
\makecell[l] {naming} &
\makecell[l] {Edge.java \\ : 3} &
\makecell[l] {为类编写文档} \\

\hline
4 & 
\makecell[l] {布尔型返回值 \\ 方法命名错误} & 
\makecell[l] {naming} &
\makecell[l] {Edge.java \\ : 33} &
\makecell[l] {用is、has、can等 \\ 命名此类方法} \\

\hline
3 & 
\makecell[l] {没有'\{'的if语句} & 
\makecell[l] {braces} &
\makecell[l] {Edge.java \\ : 46} &
\makecell[l] {为if结构增加'\{'} \\

\hline
3 & 
\makecell[l] {类中方法过多} & 
\makecell[l] {codesize} &
\makecell[l] {TextMaker.java \\ : 10} &
\makecell[l] {方法过多的类 \\ 应该重构} \\

\hline
3 & 
\makecell[l] {控制流程语句 \\ 过于复杂} & 
\makecell[l] {codesize} &
\makecell[l] {MainPage.java \\ : 69} &
\makecell[l] {重构以较少控制分支} \\

\hline
2 & 
\makecell[l] {缺少注释} & 
\makecell[l] {comments} &
\makecell[l] {Edge.java \\ : 3} &
\makecell[l] {添加注释} \\

\hline
3 & 
\makecell[l] {多个return的方法} & 
\makecell[l] {controversial} &
\makecell[l] {Edge.java \\ : 46} &
\makecell[l] {将返回值复制给变量 \\ 使用一个return返回} \\

\hline
3 & 
\makecell[l] {硬编码字面量} & 
\makecell[l] {controversial} &
\makecell[l] {Edge.java \\ : 46} &
\makecell[l] {赋予有意义的常量名} \\

\hline
3 & 
\makecell[l] {发现final局部变量} & 
\makecell[l] {controversial} &
\makecell[l] {Graph.java \\ : 51} &
\makecell[l] {将final局部变量 \\ 写成类的域} \\

\hline
3 & 
\makecell[l] {在操作数中赋值} & 
\makecell[l] {controversial} &
\makecell[l] {Graphviz.java \\ : 317} &
\makecell[l] {拆成多行,单独赋值} \\

\hline
3 & 
\makecell[l] {应显式指定访问权限} & 
\makecell[l] {controversial} &
\makecell[l] {MianPage.java \\ : 13} &
\makecell[l] {显示指定访问权限} \\

\hline
3 & 
\makecell[l] {使用具体实现的类 \\ 限制了功能的实现} & 
\makecell[l] {coupling} &
\makecell[l] {Graph.java \\ : 13} &
\makecell[l] {使用接口名 \\ 定义对象引用} \\
\hline
\end{tabular}
\end{adjustwidth}

~\\

\begin{adjustwidth}{-2em}{}
\begin{tabular}{|c|c|c|c|c|}
\hline
优先级 & 问题描述 & 违反的规则集合 & 代码行号 & 修改策略 \\
\hline
3 & 
\makecell[l] {只在初始化时赋值的 \\ 变量应声明为final} & 
\makecell[l] {design} &
\makecell[l] {Edge.java \\ : 4} &
\makecell[l] {在域中初始化 \\ 并声明为final} \\

\hline
3 & 
\makecell[l] {使用了size=0 \\ 判断集合是否为空} & 
\makecell[l] {design} &
\makecell[l] {Graph.java \\ : 155} &
\makecell[l] {使用isEmpty方法替代} \\

\hline
3 & 
\makecell[l] {发现God Class \\ (过于复杂的类)} & 
\makecell[l] {design} &
\makecell[l] {Graph.java \\ : 1} &
\makecell[l] {重构类} \\

\hline
3 & 
\makecell[l] {使用了'=' \\ 比较对象} & 
\makecell[l] {design} &
\makecell[l] {MainPage.java \\ : 74} &
\makecell[l] {替换为equals方法} \\

\hline
2 & 
\makecell[l] {发现空的catch代码块} & 
\makecell[l] {empty} &
\makecell[l] {TextMaker \\ : 136} &
\makecell[l] {抛出RuntimeException \\ 或处理异常} \\

\hline
3 & 
\makecell[l] {发现调用System.Exit} & 
\makecell[l] {j2ee} &
\makecell[l] {MainPage.java \\ : 216} &
\makecell[l] {检查逻辑 \\ System.Exit \\ 使用无误} \\

\hline
3 & 
\makecell[l] {发现可以声明 \\ 为final的参数} & 
\makecell[l] {optimizations} &
\makecell[l] {Edge.java \\ : 14} &
\makecell[l] {未在方法中修改的参数 \\ 声明为final} \\

\hline
3 & 
\makecell[l] {发现重复的字面量} & 
\makecell[l] {string} &
\makecell[l] {Graph.java \\ : 249} &
\makecell[l] {将字面量值赋予常量} \\

\hline
3 & 
\makecell[l] {发现未使用的参数} & 
\makecell[l] {unusedcode} &
\makecell[l] {Graph.java \\ : 85} &
\makecell[l] {删除参数} \\
\hline
\end{tabular}
\end{adjustwidth}

~\\
\noindent 小结:~\\
PMD的各个规则集都要它们自己的要检查的规则~\\
如naming规则集检查命名问题,design规则集检查代码的结构逻辑~\\
要根据规则集关于其规则的描述,决定是否应该使用规则集


% -------------------------------章节分割线-------------------------------
\BiChapter{FindBugs所发现的代码问清单及原因分析}{}

\begin{adjustwidth}{-1pt}{}
\begin{tabular}{|c|c|c|c|}
\hline
问题描述 & 类型 & 所在代码行号 & 修改策略 \\
\hline
\makecell[l] {不是所有的代码路径 \\ 都关闭了流} &
\makecell[l] {IO资源释放} &
\makecell[l] {Graphviz.java \\ : 315} &
\makecell[l] {try with resource \\ 管理资源} \\

\hline
\makecell[l] {try catch 语句 \\ catch未处理} &
\makecell[l] {异常处理} &
\makecell[l] {Graphviz.java \\ : 48} &
\makecell[l] {直接抛出异常或 \\ 在方法内处理异常} \\

\hline
\makecell[l] {使用$\backslash$ n作为换行符} &
\makecell[l] {格式化符号 \\ 符合平台特性} &
\makecell[l] {Graph.java \\ : 249} &
\makecell[l] {将$\backslash$ n替换为\%n} \\

\hline
\makecell[l] {发现调用 \\ System.exit} &
\makecell[l] {System.exit \\ 用法错误} &
\makecell[l] {Graph.java \\ : 216} &
\makecell[l] {检查逻辑 \\ System.Exit \\ 使用无误} \\

\hline
\makecell[l] {BufferedWriter  \\ 未指定编码} &
\makecell[l] {对默认编码依赖} &
\makecell[l] {Graphviz.java \\ : 254} &
\makecell[l] {针对平台 \\ 指定编码参数} \\

\hline
\makecell[l] {使用硬编码 \\ 引用绝对路径} &
\makecell[l] {使用硬编码} &
\makecell[l] {TextMaker.java \\ : 268} &
\makecell[l] {使用相对路径 \\ 并赋值给常量} \\
\hline

\end{tabular}
\end{adjustwidth}

% -------------------------------章节分割线-------------------------------
\BiChapter{VisualVM性能分析结果}{}
\BiSection{执行时间的统计结果与原因分析}{}
\BiSection{内存占用的统计结果与原因分析}{}
\BiSection{代码改进之后的执行时间统计结果}{}
\BiSection{代码改进之后的内存占用统计结果}{}

% -------------------------------章节分割线-------------------------------
\BiChapter{利用Git/GitHub进行协作的过程}{}

% -------------------------------章节分割线-------------------------------
\BiChapter{评述}{}
\BiSection{对代码规范方面的评述}{}
\BiSection{对代码性能方面的评述}{}

% -------------------------------章节分割线-------------------------------
\BiChapter{计划与实际进度}{}

% -------------------------------章节分割线-------------------------------
\BiChapter{小结}{}

