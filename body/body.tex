% !Mode:: "TeX:UTF-8" 
\BiChapter{实验要求}{}

% -------------------------------章节分割线-------------------------------
\BiChapter{在IntelliJ中配置代码审查与分析工具}{}
\BiSection{Checkstyle}{}
\BiSection{PMD}{}
\BiSection{FindBugs}{}
\BiSection{VisualVM}{}

% -------------------------------章节分割线-------------------------------
\BiChapter{本次实验所评审的代码}{}

% -------------------------------章节分割线-------------------------------
\BiChapter{代码review记录}{}
\begin{tabular}{|c|c|c|c|}
\hline
 问题描述 & 类型 & 所在代码行号 &  修改方式\\
\hline
\makecell[l] {在点击退出按钮后 \\ GUI界面关闭 \\ 但程序仍在运行} & 
\makecell[l] {程序退出控制}  & 
\makecell[l] {MainPage.java \\ : 467} &
\makecell[l] {增加代码 \\ setDefaultCloseOperation \\ (WindowConstants. \\ EXIT\_ON\_CLOSE) }\\ 

\hline

\end{tabular}



% -------------------------------章节分割线-------------------------------
\BiChapter{Checkstyle所发现的代码问题清单及原因分析}{}
(使用 Sun Checks 规则)
~\\

\begin{adjustwidth}{-1pt}{}
\begin{tabular}{|c|c|c|c|c|}
\hline
编号 & 问题描述 & 类型 & 所在代码行号 & 修改策略 \\
\hline
1 &
\makecell[l] {类缺少JavaDoc} &
\makecell[l] {文档缺失} &
\makecell[l] {Edge.java \\ :3} &
\makecell[l] {补充Edge类文档} \\

\hline
2 &
\makecell[l] {大括号应位于 \\ 类、方法定义同一行} &
\makecell[l] {类、方法 \\ 定义格式} &
\makecell[l] {Edge.java \\ :4} &
\makecell[l] {将大括号放在 \\ 类、方法定义同一行} \\

\hline
3 &
\makecell[l] {','前缺少空格} &
\makecell[l] {空格格式} &
\makecell[l] {Edge.java \\ :5} &
\makecell[l] {补充空格} \\

\hline
4 &
\makecell[l] {应避免在 \\ 字表达式中赋值} &
\makecell[l] {赋值格式} &
\makecell[l] {Edge.java \\ :10} &
\makecell[l] {将赋值拆分成多行} \\

\hline
5 &
\makecell[l] {参数xxx应 \\ 定义为final的} &
\makecell[l] {只读参数、变量 \\ 用法} &
\makecell[l] {Edge.java \\ :12} &
\makecell[l] {为变量声明 \\ 增加final关键字} \\

\hline
6 &
\makecell[l] {'\{'后应换行} &
\makecell[l] {避免在一行中 \\ 定义方法} &
\makecell[l] {Edge.java \\ :33} &
\makecell[l] {为方法定义 \\ 规范换行} \\

\hline
7 &
\makecell[l] {if/else结构 \\ 必须使用大括号} &
\makecell[l] {控制结构 \\ 可读性} &
\makecell[l] {Edge.java \\ :36} &
\makecell[l] {为if结构 \\ 添加大括号} \\

\hline
8 &
\makecell[l] {数组大括号 \\ 位置错误} &
\makecell[l] {数组定义规范} &
\makecell[l] {Graph.java \\ :121} &
\makecell[l] {数组中括号 \\ 移至变量类型后} \\

\hline
9 &
\makecell[l] {xxx是一个 \\ 魔术数字} &
\makecell[l] {直接常数} &
\makecell[l] {Graph.java \\ :135} &
\makecell[l] {为数字添加常量名} \\

\hline
10 &
\makecell[l] {不应以.*形式导入xxx} &
\makecell[l] {import规范} &
\makecell[l] {MainPage.java \\ :1} &
\makecell[l] {导入具体类名} \\

\hline
11 &
\makecell[l] {xxx应为private \\ 并配置访问方法} &
\makecell[l] {访问权限规范} &
\makecell[l] {MainPage.java \\ :13} &
\makecell[l] {访问权限改为private \\ 添加访问方法} \\

\hline
12 &
\makecell[l] {名称必须匹配 \\ 表达式xxx} &
\makecell[l] {命名规范} &
\makecell[l] {MainPage.java \\ :22} &
\makecell[l] {refactor修改命名} \\

\hline
13 &
\makecell[l] {本行字符数xxx \\ 最多xxx} &
\makecell[l] {行长度规范} &
\makecell[l] {MainPage.java \\ :80} &
\makecell[l] {拆成多行} \\
\hline

\end{tabular}
\end{adjustwidth}
% -------------------------------章节分割线-------------------------------
\BiChapter{PMD所发现的代码问题清单及原因分析}{}

% -------------------------------章节分割线-------------------------------
\BiChapter{FindBugs所发现的代码问题清单及原因分析}{}

% -------------------------------章节分割线-------------------------------
\BiChapter{VisualVM性能分析结果}{}
\BiSection{执行时间的统计结果与原因分析}{}
\BiSection{内存占用的统计结果与原因分析}{}
\BiSection{代码改进之后的执行时间统计结果}{}
\BiSection{代码改进之后的内存占用统计结果}{}

% -------------------------------章节分割线-------------------------------
\BiChapter{利用Git/GitHub进行协作的过程}{}

% -------------------------------章节分割线-------------------------------
\BiChapter{评述}{}
\BiSection{对代码规范方面的评述}{}
\BiSection{对代码性能方面的评述}{}

% -------------------------------章节分割线-------------------------------
\BiChapter{计划与实际进度}{}

% -------------------------------章节分割线-------------------------------
\BiChapter{小结}{}

