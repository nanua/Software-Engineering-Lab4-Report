\begin{longtable}{|X|X|X|X|}  
\hline
\textbf{任务名称} & \textbf{计划时间长度(分钟)} & \textbf{实际耗费时间(分钟)} & \textbf{提前或延期原因分析} \\
\hline
\endhead
在IntelliJ下配置工具 & 30 & 60 & IntelliJ插件下载出现异常 \\
\hline
在GitHub上找到待评审代码并fork & 10 & 10 & \\
\hline
将远程仓库clone至本地 & 5 & 5 & \\
\hline
进行代码走查并消除问题 & 120 & 240 & 在使用时发现代码错误较多,花了一定时间上网查询解决方案与修改 \\
\hline
提交代码至本地Git仓库 & 5 & 5 & \\
\hline
使用Checkstyle检查并修正代码风格 & 60 & 120 & 代码风格错误较多,花了较多时间修复 \\
\hline
使用PMD检查并修正代码错误 & 120 & 240 & PMD发现的错误较多,花了较多时间修复 \\
\hline
使用FindBugs检查并修正代码错误 & 120 & 60 & FindBugs发现的代码错误较少 \\
\hline
使用VisualVM测试并提高代码性能 & 360 & 300 & 未通过VisualVM发现严重影响代码性能的缺陷 \\
\hline
提交代码至本地Git仓库 & 5 & 5 & \\
\hline
将本地仓库push至本组远程仓库 & 5 & 1 & Git使用较熟练\\
\hline
\end{longtable}   
%%% Local Variables:
%%% mode: latex
%%% TeX-master: "../main"
%%% End:
