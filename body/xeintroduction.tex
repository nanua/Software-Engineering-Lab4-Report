% !Mode:: "TeX:UTF-8" 

\BiChapter{绪论}{Introduction}

\BiSection{XeLaTeX编译方法的配置}{Introduction to the \XeLaTeX way of compiling}

模版无法用\XeLaTeX 编译的原因主要是~\href{http://bay.uchicago.edu/tex-archive/macros/xetex/latex/xecjk/xeCJK.pdf}{xeCJK}包不兼容CJK包。
例如:xeCJK原来重写了CJK包大部分命令,如\textbackslash xeCJKcaption 。
v3.2.11之后,这个命令也废弃了。

ctexbook 中含有默认的字体设置,texlive2014-full (ubuntu 14.04) 中字体设置在下面这个文件中:
\begin{lstlisting}
/usr/share/texlive/texmf-dist/tex/latex/ctex/fontset/ctex-xecjk-winfonts.def
\end{lstlisting}

该文件中有些字体默认的设置会导致字体出错,例如无法找到[SimKai]之类。解决方法也很简单,要么在这个文件里面设置好字体;
要么在setup/package.tex中自定义好字体。
例如:ctex-xecjk-winfonts.def 文件设置为:

\begin{lstlisting}
% ctex-xecjk-winfonts.def: Windows 的 xeCJK 字体设置,默认为六种中易字体
% vim:ft=tex

\setCJKmainfont{SimSun}
\setCJKsansfont{SimHei}

\setCJKfamilyfont{zhsong}{SimSun}
\setCJKfamilyfont{zhhei}{SimHei}

\newcommand*{\songti}{\CJKfamily{zhsong}} % 宋体
\newcommand*{\heiti}{\CJKfamily{zhhei}}   % 黑体

\endinput
\end{lstlisting}

package.tex 中字体定义为:
\begin{lstlisting}
57 \def\atempxetex{xelatex}\ifx\atempxetex\usewhat %\def\atempxetex{xelatex} main.tex中已定义;
58 \usepackage[xetex,
59             bookmarksnumbered=true,
60             bookmarksopen=true,
61             colorlinks=false,
62             pdfborder={0 0 1},
63             citecolor=blue,
64             linkcolor=red,
65             anchorcolor=green,
66             urlcolor=blue,
67             breaklinks=true,
68             naturalnames  %与algorithm2e宏包协调
69             ]{hyperref}
70 %\usepackage[BoldFont,normalindentfirst,BoldFont,SlantFont]{xeCJK}
71 \defaultfontfeatures{Mapping=tex-text}
72 \xeCJKsetemboldenfactor{1}%只对随后定义的CJK字体有效
73 \setCJKfamilyfont{hei}{SimHei}
74 \xeCJKsetemboldenfactor{4}
75 \setCJKfamilyfont{song}{SimSun}
76 \xeCJKsetemboldenfactor{1}
77 \setCJKfamilyfont{fs}{FangSong}
78 \setCJKfamilyfont{kai}{KaiTi}
79 \setCJKfamilyfont{li}{LiSu}
80 \setCJKfamilyfont{xw}{STXinwei}
81 \setCJKmainfont{SimSun}
82 \setmainfont{Times New Roman}
83 \setsansfont{Arial}
84 \newcommand{\hei}{\CJKfamily{hei}}% 黑体   (Windows自带simhei.ttf)
85 \newcommand{\song}{\CJKfamily{song}}    % 宋体   (Windows自带simsun.ttf)
86 \newcommand{\fs}{\CJKfamily{fs}}        % 仿宋体 (Windows自带simfs.ttf)
87 \newcommand{\kaishu}{\CJKfamily{kai}}      % 楷体   (Windows自带simkai.ttf)
88 \newcommand{\li}{\CJKfamily{li}}        % 隶书   (Windows自带simli.ttf)
89 \newcommand{\xw}{\CJKfamily{xw}}        % 隶书   (Windows自带simli.ttf)
90 \newfontfamily\arial{Arial}                                                                                                                                               
91 \newfontfamily\timesnewroman{Times New Roman}
92 \fi
\end{lstlisting}


\BiSection{模板的使用方法介绍}{Introduction to the application method of the template}
如果您是~\LaTeX~新手,请您在使用此模板之前,先观看一下模板的使用说明~PPT 演示文档《哈工大学位论文~\LaTeX~模板使用方法介绍》,先大致了解一下此模板的使用方法,之后再准备使用此模板撰写学位论文;如果您有一定的~\LaTeX~技术基础,可以跳过此步骤。

\BiSection{哈尔滨工业大学~\LaTeX~技术交流~QQ 群介绍}{Introduction to the QQ groups for \LaTeX~technical exchange in HIT}
《哈工大硕博学位论文~\LaTeX~模板》项目现已加入哈工大研究生“学术桥”活动中,其两个官方~QQ 群分别为

\centerline{学术桥-\LaTeX~交流群~1:38872389}
\centerline{学术桥-\LaTeX~交流群~2:88984107}
\noindent 欢迎大家加入。加入~QQ 群之后,请大家将自己的名字前面加上当前月份标识,如果您没有标识,在~QQ 群人数已满但仍有人要加入此~QQ 群时,我们会将您优先请出~QQ 群,谢谢合作!您可以在~QQ 群中和其他人讨论关于此模板或其它~\LaTeX~技术相关的任何问题,在提问之前,可以先去~\href{http://bbs.ctex.org/}{CTEX 论坛}或其它网站搜索您所要得到的解决方案,然后确定是否要继续提问,从而节省您的宝贵时间。群共享中包含有大量的~\LaTeX~技术资料,方便大家下载阅读,同时还包含了最新的《哈工大硕博学位论文~\LaTeX~模板》。

原则上,模板在~Google Code 的~\href{http://code.google.com/p/plutothesis/}{PlutoThesis 项目}的\href{http://code.google.com/p/plutothesis/downloads/list}{下载列表}中也进行了同步更新,但是模板维护人员不再解答用户在此网站提出的问题,如有问题请加入上述的两个~QQ 群中再询问,敬请谅解。
